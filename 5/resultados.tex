\chapter{Conclusiones y Recomendaciones}

A través de toda esta investigación, se ha ido explorando las necesidades y posibles métodos de solución en la problemática del prediagnóstico de nódulos tiroideos a través de imágenes de ultrasonido. Lo más resaltante y constantemente repetido por otros investigadores es la alta urgencia que hay por conocer si un paciente posee un nódulo de carácter benigno o maligno, ya que a través de esta información, los especialistas pueden tomar mejores y más rápida decisiones que pueden mejorar la calidad de vida de las personas. Antes esto, se ha visto el alto desarrollo de Sistema de Diagnóstico Asistido capaces de ayudar y agilizar el proceso de diagnóstico realizado por los médicos. De entre los distintos tipos de este sistema, los actualmente de mayor desempeño son aquellos basados en algoritmos de Inteligencia Artificial, específicamente aquellos que trabajan con imágenes como lo son los CNN. Es por este motivo que en la presente investigación se ha desarrollado, a través de distintos algoritmos de Deep Learning, una herramientas capaz de acelerar el proceso de diagnóstico. 

Para ello, fue necesario determinar en primer lugar aquellas características que el conjunto de datos a usar para entrenar los algoritmos debió poseer para lograr un modelo capaz de generalizar en todas de imágenes de ultrasonido de nódulos en la tiroides. Es así que se encontró la necesidad de que el conjunto de datos tengan una cantidad de datos elevada para ambas clases de nódulos, algo que fue difícil encontrar debido a que la mayoría de conjuntos de datos de este tipo no son de libre acceso. Otra característica importante fue la validez de las etiquetas de las imágenes, ya que muchas veces se puede encontrar inconsistencias entre datos de distintos tipos que pueden definir si un nódulo es benigno o maligno. Finalmente, fue necesario encontrar datos que vengan de distintas fuentes; es decir, las imágenes de ultrasonido deben ser obtenidas a través de distintos dispositivos, esto para lograr una mayor capacidad de generalización de los modelos.

De igual manera, determinar aquellas técnicas de preprocesamiento de imágenes que deben ser aplicadas fue de gran utilidad ya que debido a la naturaleza y características que permiten el diagnóstico de un nódulo a través de imágenes de ultrasonido, no permitieron utilizar las técnicas convencionales de Aumento de Datos. En primer lugar se encontró que el clásico preprocesamiento de imágenes de reducción de dimensiones sí fue necesario debido a que esto reduce la necesidad de mayor capacidad computacional, de igual forma, por el mismo motivo, se aplicó la Normalización de las imágenes. En segundo lugar, el Aumento de Datos clásico aplicando transformaciones específicas de forma aleatoria para generar nuevos datos y lograr un balance de clases en los datos no se pudo aplicar en las imágenes de ultrasonido, esto debido a que las características que permiten determinar si un nódulo es benigno o maligno se verían totalmente alteradas si algunos de estos cambios fuese a aplicarse. Es por este motivo que se optó por una nueva forma de Aumento de Datos a través del modelo DCGAN capaz de generar imágenes falsas a través de un proceso de aprendizaje tomando como base ruido aleatorio.

Para evaluar de forma correcta la capacidad de los modelos a ser entrenados, se observó las investigaciones previas y el uso de aquellas métricas de evaluación de rendimiento usadas. Se determinó que el Accuracy es la métrica principal que describe si el modelo evaluado tiene la capacidad de clasificar correctamente si un nódulo es benigno y maligno, por lo que se decidió por su uso en esta investigación. Además, para lograr evaluar la capacidad de un modelo en clasificar correctamente si un nódulo es maligno específicamente, se optó por las métricas de Recall y Precision, usadas en gran medida también por otros investigadores, esto debido a la mayor importancia que se da por determinar si un nódulo pertenece a esta clase.

Determinar las arquitecturas de Deep Learning mayormente usadas y de alto desempeño también fue de gran importancia para reducir la cantidad de pruebas y entrenamiento de distintos algoritmos. Para lograr esto, se extrajo aquellos modelos con altas métricas de otras investigaciones desarrolladas en el mismo tipo de problemas. Así se tuvo, en primera instancia, a la arquitectura de VGG16, ResNet50 y Vision Transformer Base 16. En segundo lugar, se seleccionó también nuevas arquitecturas híbridas (ViT + CNN) que demostraban poseer un alto rendimiento comparado con los demás modelos de Deep Learning. Con todas las arquitecturas ya seleccionadas, el entrenamiento de todos estos fue más enfocado y pudo lograrse los mejores resultados posibles.

A través de todo este proceso de entrenamiento y modelos de Deep Learning con distintas técnicas, y luego de un minucioso análisis de las métricas de rendimiento, se obtuvo finalmente un modelo capaz de predecir con una precisión de 77.20\% si un nódulo es benigno o maligno a través de las imágenes de ultrasonido de la glándula tiroidal.

En esta investigación se ha encontrado con el problema de la baja cantidad de conjuntos de datos de acceso libre y con características ideales que permitan un correcto desarrollo de modelos de Deep Learning capaz de ayudar al pre diagnóstico de nódulos en la tiroides. Es por ello que, para futuras investigaciones, se recomienda una propia recolección de datos en las pertinentes instituciones, de igual forma a cómo se realiza en la mayoría de los antecedentes presentados en esta investigación. Si se consigue obtener un nuevo conjunto de datos, se podría lograr entrenar el mejor modelo DCGAN que sea capaz de generar mejores imágenes falsas, aumentando así la cantidad final de datos que se pueden tener y mejorando así las capacidades de los modelos.

El proceso de clasificación de imágenes es solo una pequeña parte de lo que puede permitir hacer un Sistema de Diagnóstico Asistido. En futuras investigaciones, se podría incluir la clasificación junto con otras tareas como la segmentación de nódulos en imágenes y el análisis del nivel de hormonas en sangre para mejorar la capacidad del sistema en determinar si un nódulo tiroideo es benigno o maligno, obteniendo así una potente herramienta capaz de ayudar en la toma de decisiones final de los médicos especialistas y así lograr una mejora en la calidad de vida de sus pacientes.

